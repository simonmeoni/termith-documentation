% !TEX root = termithDocumentation.tex
\section{dépendances externes requises}

il faut s'assurer que les programmes suivants soient installés :
\begin{itemize}
  \item \href{http://www.jason-french.com/blog/2013/03/11/installing-r-in-linux/}{R}
  \item \href{http://www.cis.uni-muenchen.de/~schmid/tools/TreeTagger/}{Treetagger}
\end{itemize}

\section{comment installé termith ?}

\begin{enumerate}
  \item s'assurer que TreeTagger, R et java 8 soient bien installés
  \item installer les librairie R suivantes : data.table et Rserve
  \item pour installer Rserve :
    \begin{enumerate}
      \item ouvrir un terminal et lancer la commande ci-dessous :
      \begin{lstlisting}[language=bash]
      $ R
      \end{lstlisting}
      \item dans l'interpréteur de commande R, taper la commande ci-dessous et suivre les instructions :
      \begin{lstlisting}[language=bash]
      $ install.packages('Rserve')
      \end{lstlisting}
      \item taper ensuite la commande ci-dessous pour charger la librairie précédemment installé :
      \begin{lstlisting}[language=bash]
      $ library('Rserve')
      \end{lstlisting}
      \item pour lancer le serveur R, taper la commande ci-dessous. Un sous-processus est alors lancé en tâche de fond, le terminal peut être fermé.
      \begin{lstlisting}[language=bash]
      $ Rserve()
      \end{lstlisting}
      \item pour installer data.table, il faut lancer une instance de R (comme ci-dessus) et lancer la commande :
      \begin{lstlisting}[language=bash]
      $ install.packages('data.table')
      \end{lstlisting}
    \end{enumerate}
\end{enumerate}
